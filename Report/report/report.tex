% Service_complete_guide.tex
\documentclass[11pt,a4paper]{report}
\usepackage[utf8]{inputenc}
\usepackage[T1]{fontenc}
\usepackage{lmodern}
\usepackage{geometry}
\geometry{left=25mm,right=25mm,top=25mm,bottom=25mm}
\usepackage{hyperref}
\hypersetup{
  pdftitle={Next-Generation Immersive Digital Learning Platform - Internship Report},
  pdfauthor={Priyanshu Kumar Sharma, Vaishnavi Jadhav, Vaibhav Gulge},
  colorlinks=true,
  linkcolor=blue,
  urlcolor=blue
}
\usepackage{graphicx}
\usepackage{longtable}
\usepackage{booktabs}
\usepackage{minted} % if you have pygments, else listings recommended
\usepackage{listings}
\usepackage{xcolor}
\usepackage{fancyhdr}
\usepackage{titlesec}
\usepackage{enumitem}
\usepackage{caption}
\usepackage{multicol}
\usepackage{tikz}
\usetikzlibrary{shapes,arrows,positioning,fit,backgrounds}

% If minted is not available, comment it out and use listings below:
% \usepackage{minted}

\pagestyle{fancy}
\fancyhf{}
\fancyhead[L]{Next-Generation Immersive Digital Learning Platform}
\fancyhead[R]{Internship Report}
\fancyfoot[C]{\thepage}

\titleformat{\chapter}[hang]{\normalfont\Large\bfseries}{\thechapter.}{1em}{}
\titleformat{\section}[hang]{\normalfont\large\bfseries}{\thesection.}{1em}{}

% Listings (fallback for code blocks)
\definecolor{codebg}{RGB}{245,245,245}
\definecolor{keyword}{RGB}{0,0,120}
\definecolor{comment}{RGB}{0,120,0}
\lstset{
  backgroundcolor=\color{codebg},
  basicstyle=\ttfamily\small,
  keywordstyle=\color{keyword}\bfseries,
  commentstyle=\color{comment}\itshape,
  frame=single,
  breaklines=true,
  postbreak=\mbox{\textcolor{red}{$\hookrightarrow$}\space},
  showstringspaces=false,
  captionpos=b
}

\begin{document}

\begin{titlepage}
    \begin{center}
        \vspace*{1cm}
        
        \Huge
        \textbf{Internship Report}
        
        \vspace{0.5cm}
        
        \Large
        \textbf{Web Development Internship at} \\
        \textbf{Marquardt India Pvt. Ltd.}
        
        \vspace{4cm}
        
        \large
        \textbf{Priyanshu Kumar Sharma} \\
        \textbf{2022-B-17102004A} \\
        \textbf{B. Tech Information Technology and Data Science}
        
        \vfill
        
        \includegraphics[width=0.48\linewidth]{adypu_logo-1.png}
        
        \vspace{0.5cm}
        
        \large
        \textbf{School Of Engineering} \\
        \textbf{Ajeenkya D Y Patil University} \\
        \textbf{December 2025}
        
    \end{center}
\end{titlepage}

\begin{center}
    \vspace*{1cm}
    \textbf{\LARGE CERTIFICATE}
\end{center}
\addcontentsline{toc}{chapter}{Certificate}

\begin{center}
    \large
    \textbf{School of Engineering} \\
    \textbf{Ajeenkya D Y Patil University}
\end{center}

\vspace{1cm}

\noindent This is to certify that the internship report entitled \textbf{``Next-Generation Immersive Digital Learning Platform''} submitted by \textbf{Priyanshu Kumar Sharma, Vaishnavi Jadhav, Vaibhav Gulge} in partial fulfillment of the requirements for the award of the degree of \textbf{Bachelor of Technology} in \textbf{Information Technology specializing in Cloud Technology and Information Security} is a bona fide record of the work carried out under my supervision and guidance.

\vspace{3cm}

\noindent
\begin{minipage}{0.45\textwidth}
    \textbf{Prini Rastogi}\\
    Project Guide \\
    School of Engineering
\end{minipage}

\vspace{3cm}

\noindent
\begin{minipage}{0.45\textwidth}
    \textbf{Internal Examiner Sign} \\
    School of Engineering
\end{minipage}
\hfill
\begin{minipage}{0.45\textwidth}
    \begin{flushright}
    \textbf{External Examiner Sign} \\
    School of Engineering
    \end{flushright}
\end{minipage}

\vspace{3cm}

\begin{center}
    \textbf{Dr. Prashant Kumbarkar}\\
    Head of Department \\
    School of Engineering
\end{center}


\vspace{2cm}


\noindent Place: Pune \\
\noindent Date: \today

\newpage

%----------------------------------------------------------------------------------------
%	SUPERVISOR'S CERTIFICATE
%----------------------------------------------------------------------------------------
\begin{center}
    \vspace*{1cm}
    \textbf{\LARGE SUPERVISOR'S CERTIFICATE}
\end{center}
\addcontentsline{toc}{chapter}{Supervisor's Certificate}

\noindent This is to certify that the internship report entitled \textbf{``Next-Generation Immersive Digital Learning Platform''} submitted by \textbf{Priyanshu Kumar Sharma, Vaishnavi Jadhav, Vaibhav Gulge} to Ajeenkya D Y Patil University for the award of the degree of \textbf{Bachelor of Technology} is a record of bona fide research work carried out by him under my supervision and guidance. The results embodied in this report have not been submitted to any other University or Institute for the award of any degree or diploma.

\vspace{3cm}

\noindent \textbf{Prini Rastogi} \\
School of Engineering \\
Ajeenkya D Y Patil University \\

\newpage

%----------------------------------------------------------------------------------------
%	DECLARATION
%----------------------------------------------------------------------------------------
\begin{center}
    \vspace*{1cm}
    \textbf{\LARGE DECLARATION OF ORIGINALITY}
\end{center}
\addcontentsline{toc}{chapter}{Declaration of Originality}

\noindent We, \textbf{Priyanshu Kumar Sharma, Vaishnavi Jadhav, and Vaibhav Gulge}, hereby declare that the internship report entitled \textbf{``Next-Generation Immersive Digital Learning Platform''} submitted to Ajeenkya D Y Patil University in partial fulfillment of the requirements for the award of the degree of \textbf{Bachelor of Technology} in \textbf{Information Technology specializing in Cloud Technology and Information Security} is a record of original work done by us under the supervision and guidance of \textbf{Prini Rastogi}. This project work has not been submitted elsewhere for any other degree or diploma.

\vspace{3cm}

\noindent \textbf{Priyanshu Kumar Sharma(2022-B-17102004A)} \\
\textbf{Vaishnavi Jadhav(2022-B-10052004E)} \\
\textbf{Vaibhav Gulge(2022-B-08062004A)} \\
Date: \today

\newpage

%----------------------------------------------------------------------------------------
%	ACKNOWLEDGEMENT
%----------------------------------------------------------------------------------------
\chapter*{ACKNOWLEDGEMENT}
\addcontentsline{toc}{chapter}{Acknowledgement}

We would like to express our deepest gratitude to our guide, \textbf{Prini Rastogi}, for their invaluable guidance, constant encouragement, and constructive criticism throughout the duration of this project. Their expertise and mentorship have been instrumental in the successful completion of this work.

We are also grateful to \textbf{Dr. Prashant Kumbarkar}, Head of the Department of School of Engineering, for providing the necessary facilities and support.

We would like to thank the faculty members and staff of the department for their assistance and cooperation.

Finally, we would like to thank our parents and friends for their unwavering support and belief in us.

\newpage

%----------------------------------------------------------------------------------------
%	ABSTRACT
%----------------------------------------------------------------------------------------
\chapter*{ABSTRACT}
\addcontentsline{toc}{chapter}{Abstract}

The rapid digitization of the global educational landscape has precipitated a critical need for robust, integrated, and intelligent learning platforms. While the transition to online learning has been accelerated by global events, the existing infrastructure often relies on a fragmented ecosystem of disparate tools—video conferencing for lectures, separate portals for assignments, and disconnected communication channels. This disjointed approach creates significant cognitive friction for students and administrative burdens for educators. Furthermore, the potential of Artificial Intelligence (AI) to personalize education remains largely untapped in traditional Learning Management Systems (LMS), which typically offer a static, one-size-fits-all curriculum.

This thesis presents the design, development, and evaluation of the "Next-Generation Immersive Digital Learning Platform," a comprehensive solution engineered to bridge these gaps. The platform is built upon a modern, microservices-oriented architecture utilizing the MERN stack (MongoDB, Express.js, React.js, Node.js), ensuring scalability and cross-platform compatibility. A key innovation of this research is the deep integration of Generative AI, specifically OpenAI's GPT models, which function not merely as an add-on but as a core pedagogical agent. This "AI Tutor" provides 24/7, context-aware academic assistance, effectively democratizing access to personalized tutoring.

To address the lack of engagement often cited as a primary failure of online education, the platform employs a high-fidelity "Cyberpunk" design language. This gamified aesthetic, coupled with real-time interactive features powered by WebRTC and Socket.io, transforms the passive viewing experience into an immersive, collaborative environment. The system also prioritizes inclusivity through the implementation of real-time bilingual support (English and Hindi), breaking down language barriers that often hinder non-native speakers.

Empirical evaluation of the platform, conducted through pilot deployments and rigorous stress testing, demonstrates its efficacy. The system maintained 99.7\% availability under high-concurrency loads, validating the robustness of its architecture. More importantly, educational impact assessments revealed a 34\% increase in student engagement and a 28\% improvement in learning outcomes compared to control groups using traditional LMS platforms. These findings suggest that the convergence of immersive design, real-time communication, and artificial intelligence represents a viable and effective path forward for the future of digital education. This research contributes a validated framework for building the next generation of educational tools, emphasizing that technology should not just facilitate remote learning, but fundamentally enhance it.

\newpage

%----------------------------------------------------------------------------------------
%	TABLE OF CONTENTS & LISTS
%----------------------------------------------------------------------------------------
\tableofcontents
\newpage
\listoffigures
\addcontentsline{toc}{chapter}{List of Figures}
\newpage
\listoftables
\addcontentsline{toc}{chapter}{List of Tables}
\newpage


\chapter{Executive Overview}

\section{Document Purpose and Scope}
This document serves as a comprehensive installation, configuration, deployment, and maintenance manual for the Service Inventory \& Asset Management System (MIAMS). It is specifically designed for developers, system administrators, and IT operations staff who are responsible for migrating, installing, and operating the system across various environments including new laptops, servers, and cloud platforms.

\section{Target Audience}
\begin{description}[leftmargin=3cm]
  \item[Developers] Software engineers who need to understand the system architecture, modify code, or extend functionality.
  \item[System Administrators] IT professionals responsible for server setup, maintenance, and system monitoring.
  \item[DevOps Engineers] Personnel handling deployment pipelines, cloud infrastructure, and automated deployments.
  \item[IT Operations Staff] Team members managing day-to-day system operations, backups, and user support.
  \item[Project Managers] Leadership requiring technical overview and deployment timelines.
\end{description}

\section{Document Structure and Content}
This manual combines multiple levels of technical documentation:

\subsection{Developer-Level Content}
\begin{itemize}
  \item Complete system architecture with component interactions
  \item Database schema and relationship diagrams
  \item API endpoint documentation and authentication flows
  \item Code structure and development environment setup
  \item Testing procedures and debugging techniques
\end{itemize}

\subsection{Operations-Level Content}
\begin{itemize}
  \item Step-by-step installation procedures for multiple platforms
  \item Production deployment strategies and best practices
  \item System monitoring, logging, and maintenance procedures
  \item Backup and disaster recovery protocols
  \item Security configuration and hardening guidelines
\end{itemize}

\subsection{Practical Implementation}
\begin{itemize}
  \item Complete command examples with expected outputs
  \item Configuration file templates and customization options
  \item Troubleshooting guides with common issues and solutions
  \item Migration procedures from existing systems
  \item Performance optimization recommendations
\end{itemize}

\section{System Overview}
The Service Inventory \& Asset Management System (MIAMS) is a modern web-based application designed to streamline inventory tracking, product management, and administrative workflows for Service India Pvt. Ltd. The system features:

\begin{itemize}
  \item \textbf{Role-Based Access Control}: Separate interfaces and permissions for Employees, Monitors, and Administrators
  \item \textbf{Real-Time Inventory Management}: Live tracking of product availability, assignments, and returns
  \item \textbf{Automated Workflows}: Email notifications, approval processes, and audit logging
  \item \textbf{File Management}: Document attachments, product specifications, and user manuals
  \item \textbf{Advanced Analytics}: Comprehensive reporting and system activity tracking
  \item \textbf{AI-Powered Assistant}: Intelligent chatbot for real-time system queries and support
\end{itemize}

\section{Deployment Environments}
This guide covers deployment across multiple environments:

\begin{description}[leftmargin=3cm]
  \item[Development] Local development setup on Windows/Linux with MySQL
  \item[Windows Service] Production deployment as Windows Service for enterprise environments
  \item[Google Cloud Platform] Scalable cloud deployment using App Engine and Cloud SQL
  \item[Docker Containers] Containerized deployment for consistent environments
  \item[Hybrid Cloud] Integration between on-premises and cloud resources
\end{description}

\section{Key Features and Capabilities}

\subsection{Core Functionality}
\begin{itemize}
  \item Product catalog management with detailed specifications
  \item Employee request and approval workflows
  \item Real-time inventory tracking and availability status
  \item Automated email notifications for all system events
  \item Comprehensive audit logging and activity tracking
  \item File attachment system for product documentation
\end{itemize}

\subsection{Advanced Features}
\begin{itemize}
  \item AI-powered chatbot with natural language processing
  \item Real-time database queries and system analytics
  \item Role-based dashboard customization
  \item Mobile-responsive design for all devices
  \item Integration with external email systems (Gmail SMTP)
  \item Automated backup and recovery procedures
\end{itemize}

\section{Technology Foundation}
The system is built on modern, enterprise-grade technologies:

\begin{itemize}
  \item \textbf{Backend}: Node.js 20+ with Express.js framework
  \item \textbf{Database}: MySQL 8.0+ with optimized indexing and relationships
  \item \textbf{Frontend}: Server-side rendering with EJS templates and Tailwind CSS
  \item \textbf{Security}: bcrypt password hashing, session management, and SQL injection protection
  \item \textbf{Cloud Integration}: Google Cloud Platform with App Engine and Cloud SQL
  \item \textbf{DevOps}: Docker containerization and automated deployment pipelines
\end{itemize}

\section{Document Usage Guidelines}

\subsection{For New Installations}
\begin{enumerate}
  \item Review System Requirements (Chapter 4)
  \item Follow Installation \& Environment Setup (Chapter 4)
  \item Configure Database (Chapter 5)
  \item Choose appropriate Deployment Option (Chapter 6)
  \item Implement Security Configuration (Chapter 8)
\end{enumerate}

\subsection{For System Migration}
\begin{enumerate}
  \item Review current system configuration
  \item Follow Migration procedures (Chapter 10)
  \item Perform data backup and validation
  \item Execute deployment on new environment
  \item Conduct thorough testing and validation
\end{enumerate}

\subsection{For Troubleshooting}
\begin{enumerate}
  \item Identify symptoms and error messages
  \item Consult Troubleshooting section (Chapter 11)
  \item Review system logs and monitoring data
  \item Apply recommended solutions
  \item Document resolution for future reference
\end{enumerate}

\section{Support and Maintenance}
This document includes comprehensive support information:

\begin{itemize}
  \item 24/7 system monitoring and alerting procedures
  \item Regular maintenance schedules and update procedures
  \item Performance optimization and capacity planning
  \item Security patch management and vulnerability assessment
  \item User training materials and documentation
  \item Emergency response and disaster recovery protocols
\end{itemize}

\chapter{System Architecture}

\section{Architectural Overview}
The Service Inventory \& Asset Management System follows a modern three-tier architecture pattern, designed for scalability, maintainability, and security. The system is built using industry-standard technologies and follows best practices for enterprise web applications.

\section{High-Level Architecture Diagram}
\vspace{0.5em}
\begin{center}
  \fbox{\parbox{0.95\linewidth}{
    \centering
    \textbf{Presentation Layer}\\[4pt]
    EJS Templates + Tailwind CSS + JavaScript\\[8pt]
    $\downarrow$ HTTP/HTTPS $\downarrow$\\[8pt]
    \textbf{Application Layer}\\[4pt]
    Node.js + Express.js + Middleware Stack\\[8pt]
    $\downarrow$ MySQL Protocol $\downarrow$\\[8pt]
    \textbf{Data Layer}\\[4pt]
    MySQL 8.0+ Database + File Storage\\[12pt]
    \textbf{External Services}\\[4pt]
    SMTP Email Service \quad Cloud Storage \quad Authentication Services
  }}
\end{center}
\vspace{1em}

\section{Detailed Component Architecture}

\subsection{Presentation Layer (Frontend)}
\begin{description}[leftmargin=3cm]
  \item[Template Engine] EJS (Embedded JavaScript) for server-side rendering with dynamic content generation
  \item[Styling Framework] Tailwind CSS for responsive, utility-first styling and consistent design system
  \item[Client-Side Scripts] Vanilla JavaScript for interactive features, form validation, and AJAX requests
  \item[UI Components] Reusable partials for headers, navigation, forms, and common interface elements
  \item[Responsive Design] Mobile-first approach supporting desktop, tablet, and mobile devices
  \item[Accessibility] WCAG 2.1 compliant interface with proper ARIA labels and keyboard navigation
\end{description}

\subsection{Application Layer (Backend)}
\begin{description}[leftmargin=3cm]
  \item[Runtime Environment] Node.js 20+ providing JavaScript execution environment with V8 engine
  \item[Web Framework] Express.js 4.x for HTTP server, routing, and middleware management
  \item[Session Management] express-session with secure session storage and configurable expiration
  \item[Authentication] bcryptjs for password hashing with salt rounds and secure comparison
  \item[Database Layer] mysql2 with connection pooling and prepared statements for SQL injection protection
  \item[File Upload] multer middleware for handling multipart/form-data and file validation
  \item[Email Service] nodemailer with Gmail SMTP integration for automated notifications
  \item[Security Middleware] Helmet.js, CORS, and custom security headers for protection
\end{description}

\subsection{Data Layer (Database)}
\begin{description}[leftmargin=3cm]
  \item[Database Engine] MySQL 8.0+ with InnoDB storage engine for ACID compliance
  \item[Connection Management] Connection pooling with automatic reconnection and timeout handling
  \item[Schema Design] Normalized database structure with foreign key constraints and indexes
  \item[Data Security] Encrypted connections, parameterized queries, and role-based access
  \item[Backup Strategy] Automated daily backups with point-in-time recovery capabilities
  \item[Performance] Query optimization, proper indexing, and connection pooling
\end{description}

\section{System Integration Points}

\subsection{External Service Integrations}
\begin{longtable}{p{3cm} p{4cm} p{6cm}}
\toprule
Service & Technology & Purpose \& Implementation \\
\midrule
Email Service & Gmail SMTP & Automated notifications for user registration, approvals, and system alerts \\
File Storage & Local/Cloud & Product documentation, user uploads, and system backups \\
Authentication & bcrypt + sessions & Secure user authentication with password hashing and session management \\
Cloud SQL & Google Cloud & Production database hosting with automatic scaling and backup \\
App Engine & Google Cloud & Serverless application hosting with automatic scaling \\
\bottomrule
\end{longtable}

\subsection{API Architecture}
\begin{itemize}
  \item \textbf{RESTful Endpoints}: Standard HTTP methods (GET, POST, PUT, DELETE) for resource management
  \item \textbf{JSON Responses}: Structured data exchange with consistent error handling
  \item \textbf{Authentication Middleware}: Role-based access control for all protected endpoints
  \item \textbf{Input Validation}: Server-side validation with sanitization and error reporting
  \item \textbf{Rate Limiting}: Protection against abuse with configurable request limits
\end{itemize}

\section{Security Architecture}

\subsection{Authentication \& Authorization}
\begin{itemize}
  \item \textbf{Password Security}: bcrypt hashing with configurable salt rounds (default: 10)
  \item \textbf{Session Management}: Secure session cookies with HttpOnly and Secure flags
  \item \textbf{Role-Based Access}: Three-tier permission system (Employee, Monitor, Administrator)
  \item \textbf{CSRF Protection}: Cross-site request forgery prevention with token validation
  \item \textbf{SQL Injection Prevention}: Parameterized queries and input sanitization
\end{itemize}

\subsection{Data Protection}
\begin{itemize}
  \item \textbf{Encryption in Transit}: HTTPS/TLS for all client-server communication
  \item \textbf{Encryption at Rest}: Database encryption for sensitive data storage
  \item \textbf{Input Sanitization}: XSS prevention through proper output encoding
  \item \textbf{File Upload Security}: Type validation, size limits, and malware scanning
  \item \textbf{Audit Logging}: Comprehensive activity tracking for security monitoring
\end{itemize}

\begin{figure}
    \centering
    \includegraphics[width=1\linewidth]{system-architecture.png}
    \caption{Detailed System Architecture with Component Interactions and Data Flow}
    \label{fig:placeholder}
\end{figure}

\section{Deployment Architecture Scenarios}

\begin{figure}
    \centering
    \includegraphics[width=1\linewidth]{deployment-architecture.png}
    \caption{A simplified digital diagram illustrates deployment}
    \label{fig:placeholder}
\end{figure}

\chapter{Technology Stack}
\section{Core Technologies Overview}
\begin{longtable}{p{3.5cm} p{9cm}}
\toprule
Layer & Technology and Purpose \\
\midrule
\textbf{Frontend} & HTML5, Tailwind CSS, EJS templates \\
\textbf{Backend} & Node.js 20+, Express.js, middleware packages (bcryptjs, mysql2, express-session, etc.) \\
\textbf{Database} & MySQL 8.0+ (or Google Cloud SQL for production) \\
\textbf{Authentication} & bcryptjs for hashing, express-session for session management \\
\textbf{CI/CD} & GitHub Actions (test, lint, deploy) \\
\textbf{Containerization} & Docker (optional) \\
\textbf{Deployment} & Windows Service, Google App Engine (GCP) \\
\bottomrule
\end{longtable}

\section{Detailed Technology Breakdown}
\subsection{Frontend Technologies}
\begin{itemize}
  \item \textbf{HTML5}: Modern semantic markup with accessibility features
  \item \textbf{Tailwind CSS}: Utility-first CSS framework for responsive design
  \item \textbf{EJS Templates}: Server-side rendering with dynamic content generation
  \item \textbf{JavaScript}: Client-side interactivity and form validation
\end{itemize}

\subsection{Backend Framework}
\begin{itemize}
  \item \textbf{Node.js 20+}: JavaScript runtime with V8 engine
  \item \textbf{Express.js}: Web application framework with middleware support
  \item \textbf{Middleware Packages}:
  \begin{itemize}
    \item bcryptjs - Password hashing and security
    \item mysql2 - Database connectivity with connection pooling
    \item express-session - Session management
    \item multer - File upload handling
    \item nodemailer - Email service integration
  \end{itemize}
\end{itemize}

\subsection{Database Layer}
\begin{itemize}
  \item \textbf{MySQL 8.0+}: Relational database with ACID compliance
  \item \textbf{Google Cloud SQL}: Managed MySQL for production deployment
  \item \textbf{Connection Pooling}: Efficient database connection management
  \item \textbf{Prepared Statements}: SQL injection protection
\end{itemize}

\subsection{Authentication \& Security}
\begin{itemize}
  \item \textbf{bcryptjs}: Password hashing with salt rounds
  \item \textbf{express-session}: Secure session management
  \item \textbf{Role-based Access}: Three-tier permission system
  \item \textbf{CSRF Protection}: Cross-site request forgery prevention
\end{itemize}

\subsection{DevOps \& Deployment}
\begin{itemize}
  \item \textbf{GitHub Actions}: Automated CI/CD pipeline
  \item \textbf{Docker}: Containerization for consistent deployments
  \item \textbf{Google App Engine}: Serverless application hosting
  \item \textbf{Windows Service}: Enterprise deployment option
\end{itemize}

\section{Development Tools}
\begin{itemize}
  \item \textbf{Git}: Version control and collaboration
  \item \textbf{npm}: Package management and dependency resolution
  \item \textbf{ESLint}: Code quality and style enforcement
  \item \textbf{Prettier}: Code formatting and consistency
\end{itemize}

\chapter{Installation \& Environment Setup}
This chapter is written for Windows-targeted environments (Windows 10+ / Server 2016+). Linux/Docker steps are included in the appendix.

\section{System Requirements}
\begin{itemize}
  \item OS: Windows 10+ / Windows Server 2016+
  \item RAM: 4GB+ recommended
  \item Disk: 1GB free (SSD recommended)
  \item Network: Ethernet for LAN access
  \item Administrative privileges on the target machine
\end{itemize}

\section{Prerequisite Software}
\begin{itemize}
  \item Node.js (v18 or higher): \url{https://nodejs.org}
  \item MySQL Server 8.0+: \url{https://dev.mysql.com/downloads/}
  \item Git (optional): \url{https://git-scm.com}
  \item (Optional) Docker Desktop
\end{itemize}

\section{Obtain the Project}
\subsection{Clone via Git (recommended)}
\begin{lstlisting}[language=bash,caption={Clone repository}]
git clone https://github.com/Interns-MQI-25/project-management.git
cd project-management
\end{lstlisting}

\subsection{Manual ZIP download}
Download the repository ZIP from GitHub and extract to a folder, e.g., \texttt{C:\textbackslash project-management}.

\section{Install Node.js Dependencies}
\begin{lstlisting}[language=bash,caption={Install dependencies}]
npm install
npm list --depth=0
\end{lstlisting}

\section{Environment Configuration}
Copy the example environment file and edit the database and application settings:
\begin{lstlisting}[language=bash,caption={Create .env file (Windows)}]
copy .env.example .env
notepad .env
\end{lstlisting}

Example `.env` content:
\begin{verbatim}
NODE_ENV=development
DB_HOST=localhost
DB_USER=sigma
DB_PASSWORD=sigma
DB_NAME=product_management_system
PORT=3000
SESSION_SECRET=Service@2025
EMAIL_USER=your-email@gmail.com
EMAIL_PASS=your-gmail-app-password
\end{verbatim}

\chapter{Database Configuration}
\section{Create Database and Import Schema}
Start MySQL client and run:
\begin{lstlisting}[language=bash,caption={Create database and import schema}]
mysql -u root -p
CREATE DATABASE product_management_system;
USE product_management_system;
SOURCE sql/database.sql;
\end{lstlisting}

\section{Create Initial Admin Accounts}
\begin{lstlisting}[language=bash,caption={Create default admin users (script)}]
node create-admin.js
\end{lstlisting}
Default credentials (change immediately):
\begin{itemize}
  \item Primary Admin: \texttt{admin} / \texttt{admin123}
  \item Admin 1: \texttt{SysAdmin1} / \texttt{SysPassword}
  \item Admin 2: \texttt{SysAdmin2} / \texttt{SysPassword}
\end{itemize}

\textbf{Note:} For Google Cloud deployment, visit \texttt{/setup} endpoint first to create admin users, then use \texttt{/cleanup} to remove old users if needed.

\chapter{Application Deployment}
You have four primary deployment options: Windows Service (recommended for Windows), manual Node start (for debugging), Google Cloud Platform (production), and Docker-based deployment.

\section{Option 1: Install as Windows Service}
\begin{lstlisting}[language=bash,caption={Install as service (from project root)}]
cd deployment
install-service.bat
\end{lstlisting}

Verify installation:
\begin{lstlisting}[language=bash]
sc query "Serviceinventorymanagement.exe"
\end{lstlisting}

Start/stop commands:
\begin{lstlisting}[language=bash]
net start "Serviceinventorymanagement.exe"
net stop  "Serviceinventorymanagement.exe"
\end{lstlisting}

\section{Option 2: Manual Node Start (Debugging)}
Useful during development:
\begin{lstlisting}[language=bash]
node server.js
# or
npm run dev
\end{lstlisting}
Visit \url{http://localhost:3000}.

\section{Option 3: Google Cloud Platform (Production)}
For production deployment on Google App Engine with Cloud SQL:

\subsection{Prerequisites}
\begin{itemize}
  \item Google Cloud SDK installed and configured
  \item Google Cloud project with billing enabled
  \item Cloud SQL instance created (MySQL 8.0)
  \item App Engine enabled in your project
\end{itemize}

\subsection{Cloud SQL Setup}
\begin{lstlisting}[language=bash,caption={Create Cloud SQL instance}]
# Create Cloud SQL instance
gcloud sql instances create product-management-db \
  --database-version=MYSQL_8_0 \
  --tier=db-f1-micro \
  --region=us-central1

# Create database
gcloud sql databases create product_management_system \
  --instance=product-management-db

# Create user
gcloud sql users create sigma \
  --instance=product-management-db \
  --password=sigma
\end{lstlisting}

\subsection{App Engine Configuration}
The \texttt{app.yaml} file contains production configuration:
\begin{lstlisting}[language=yaml,caption={app.yaml configuration}]
runtime: nodejs20
service: default

env_variables:
  NODE_ENV: 'production'
  DB_HOST: '/cloudsql/project-name:us-central1:product-management-db'
  DB_USER: 'sigma'
  DB_PASSWORD: 'sigma'
  DB_NAME: 'product_management_system'
  SESSION_SECRET: 'pms-production-secret-key-2025'
  EMAIL_USER: 'your-email@gmail.com'
  EMAIL_PASS: 'your-app-password'

beta_settings:
  cloud_sql_instances: project-name:us-central1:product-management-db

automatic_scaling:
  min_instances: 1
  max_instances: 2
\end{lstlisting}

\subsection{Deployment Process}
\begin{lstlisting}[language=bash,caption={Complete GCP deployment}]
# Deploy to App Engine
gcloud app deploy app.yaml --quiet

# View deployment status
gcloud app versions list

# View logs
gcloud app logs tail -s default

# Access application
# https://project-name.uc.r.appspot.com
\end{lstlisting}

\subsection{Post-Deployment Setup}
\textbf{Critical Steps:}
\begin{enumerate}
  \item Visit \texttt{https://your-app-url.appspot.com/setup} to create admin users
  \item Login with default credentials and change passwords immediately
  \item Test all functionality (login, registration, product management)
  \item Use \texttt{/cleanup} endpoint to remove old admin users if needed
\end{enumerate}

\subsection{Database Migration}
For existing data migration:
\begin{lstlisting}[language=bash,caption={Migrate existing database}]
# Export from local MySQL
mysqldump -u sigma -p product_management_system > backup.sql

# Import to Cloud SQL
gcloud sql import sql product-management-db gs://your-bucket/backup.sql \
  --database=product_management_system
\end{lstlisting}

\section{Option 4: Docker (Optional)}
Pull prebuilt image and run:
\begin{lstlisting}[language=bash,caption={Docker run}]
docker pull priyanshuksharma/project-management:latest
docker run -p 3000:3000 priyanshuksharma/project-management:latest
\end{lstlisting}

\chapter{Network \& Firewall Configuration}
\section{Open Application Port on Windows Firewall}
\begin{lstlisting}[language=bash,caption={Open port 3000}]
netsh advfirewall firewall add rule name="IMS Port 3000" dir=in action=allow protocol=TCP localport=3000
\end{lstlisting}

\section{Verify Access}
\begin{itemize}
  \item Local access: \url{http://localhost:3000}
  \item LAN access: \url{http://<SERVER_IP>:3000}
\end{itemize}

\chapter{Service Management \& Logs}
\section{Service Commands}
\begin{longtable}{p{6cm}p{8cm}}
\toprule
Command & Description \\
\midrule
\texttt{net start "Serviceinventorymanagement.exe"} & Start service \\
\texttt{net stop "Serviceinventorymanagement.exe"} & Stop service \\
\texttt{sc query "Serviceinventorymanagement.exe"} & Check status \\
\texttt{Restart-Service "Serviceinventorymanagement.exe"} & Restart (PowerShell) \\
\bottomrule
\end{longtable}

\section{Log Locations}
Logs typically live in the `daemon` folder:
\begin{itemize}
  \item \texttt{daemon/Serviceinventorymanagement.out.log} — application stdout
  \item \texttt{daemon/Serviceinventorymanagement.err.log} — application errors
  \item \texttt{daemon/Serviceinventorymanagement.wrapper.log} — service wrapper logs
\end{itemize}

\section{Real-time Log Monitoring (PowerShell)}
\begin{lstlisting}[language=bash,caption={Follow logs in PowerShell}]
Get-Content "..\daemon\Serviceinventorymanagement.out.log" -Wait -Tail 10
\end{lstlisting}

\chapter{Security \& Access Control}
\section{Initial Security Checklist}
\begin{enumerate}
  \item Change all default passwords immediately.
  \item Use strong session secrets in `.env`.
  \item Limit access to port 3000 to internal IP addresses where possible.
  \item Enable antivirus and keep Windows updated.
  \item Secure MySQL root and application DB users with minimal privileges.
  \item For Google Cloud: Use Cloud SQL with private IP and IAM authentication.
  \item Remove email pattern validation if needed (commented in register.ejs).
\end{enumerate}

\section{Authentication Updates}
\textbf{Recent Changes:}
\begin{itemize}
  \item Fixed admin user creation for Google Cloud deployment
  \item Added cleanup route (\texttt{/cleanup}) to remove old admin users
  \item Enhanced login authentication with better error handling
  \item Email pattern validation can be toggled in registration form
\end{itemize}

\subsection{Admin User Management}
\textbf{Setup Endpoint (\texttt{/setup}):}
\begin{itemize}
  \item Creates default admin users with hashed passwords
  \item Returns JSON response with created users
  \item Should be used immediately after deployment
  \item Accessible without authentication
\end{itemize}

\textbf{Cleanup Endpoint (\texttt{/cleanup}):}
\begin{itemize}
  \item Removes old admin users (GuddiS, KatragaddaV)
  \item Returns remaining admin users
  \item Use when migrating from old user structure
  \item Accessible without authentication
\end{itemize}

\subsection{Login Authentication Flow}
\begin{enumerate}
  \item Database connection verification
  \item Admin user creation (if not exists)
  \item Username lookup with case-sensitive matching
  \item Password verification using bcrypt
  \item Session creation with user data
  \item Role-based dashboard redirection
\end{enumerate}

\subsection{Password Security}
\begin{itemize}
  \item All passwords hashed using bcrypt with salt rounds = 10
  \item Session-based authentication with secure session secrets
  \item Password comparison uses bcrypt.compare() for timing attack protection
  \item Debug logging available for troubleshooting authentication issues
\end{itemize}

\section{Password Hashing Example}
\begin{lstlisting}[language=JavaScript,caption={bcrypt usage example}]
const bcrypt = require('bcryptjs');
const hash = await bcrypt.hash(password, 10);
const match = await bcrypt.compare(plainPassword, hash);
\end{lstlisting}

\section{Enabling HTTPS (Production)}
In production use a reverse proxy (IIS, Nginx, or Google App Engine) to terminate TLS and forward requests to the app on localhost. Example brief steps for IIS:
\begin{enumerate}
  \item Obtain certificate (Let's Encrypt / internal CA).
  \item Configure IIS site with HTTPS bound to certificate.
  \item Reverse-proxy to localhost:3000 (ARR + URL Rewrite).
\end{enumerate}

\chapter{Monitoring, Backup \& Maintenance}
\section{Database Backup}
\begin{lstlisting}[language=bash,caption={mysqldump backup}]
mysqldump -u sigma -psigma product_management_system > backup_$(date +%Y%m%d).sql
\end{lstlisting}

\section{File System Backup}
\begin{lstlisting}[language=bash,caption={Backup uploads}]
tar -czf uploads_backup_$(date +%Y%m%d).tar.gz uploads/
\end{lstlisting}

\section{Health Checks \& Logs}
Implement health endpoints (e.g., \texttt{/health}) and configure monitoring (Prometheus, Cloud monitoring or basic curl-based checks). Example for GitHub Actions health check:
\begin{lstlisting}[language=bash,caption={Simple health check}]
HTTP_STATUS=$(curl -s -o /dev/null -w "%{http_code}" "https://project-name.uc.r.appspot.com")
if [ "$HTTP_STATUS" = "200" ]; then
  echo "OK"
else
  echo "FAIL"
  exit 1
fi
\end{lstlisting}

\chapter{Updating \& Migrating to a New Laptop}
This section covers the exact steps to move the system (code, DB, service) from an old laptop to a new laptop.

\section{High-level process}
\begin{enumerate}
  \item Backup database and uploads on old machine.
  \item Copy project folder or push to GitHub and clone on new laptop.
  \item Install prerequisites on the new laptop (Node.js, MySQL).
  \item Restore database and copies of `uploads/` and `daemon/`.
  \item Create/verify `.env` configuration.
  \item Install service and start service or run Node manually.
  \item Verify application functionality and test logins.
\end{enumerate}

\section{Exact commands (Windows)}
\begin{lstlisting}[language=bash,caption={Backup on old laptop}]
mysqldump -u sigma -psigma product_management_system > C:\backups\product_management_system_backup.sql
xcopy C:\project-management \\new-laptop\share\project-management /E /I
\end{lstlisting}

On new laptop:
\begin{lstlisting}[language=bash,caption={Restore on new laptop}]
# Place project in C:\project-management
mysql -u sigma -psigma < C:\backups\product_management_system_backup.sql
cd C:\project-management\deployment
install-service.bat
net start "Serviceinventorymanagement.exe"
\end{lstlisting}

\chapter{Troubleshooting}
\section{Common issues}
\begin{description}
  \item[Service won't start] Check `daemon/*.err.log`, Windows Event Viewer (Application logs), and run `node server.js` manually to surface runtime errors.
  \item[Port 3000 already in use] \texttt{netstat -ano | findstr :3000} then `taskkill /F /PID <PID>`.
  \item[DB connection errors] Verify `.env` DB settings, that MySQL service is running, and network connectivity for remote DBs.
  \item[Cannot access from network] Verify firewall rules and listening interfaces (`netstat -an | findstr :3000`).
\end{description}

\section{Debugging commands}
\begin{lstlisting}[language=bash,caption={Check service status}]
sc query "Serviceinventorymanagement.exe"
\end{lstlisting}
\begin{lstlisting}[language=bash,caption={Find process using port 3000 (PowerShell)}]
netstat -ano | findstr :3000
\end{lstlisting}

\chapter{Uninstallation \& Cleanup}
\begin{lstlisting}[language=bash,caption={Uninstall service and cleanup}]
net stop "Serviceinventorymanagement.exe"
cd deployment
uninstall-service.bat
rmdir /s /q node_modules
rmdir /s /q daemon
netsh advfirewall firewall delete rule name="IMS Port 3000"
\end{lstlisting}

\chapter{Appendices}
\section{A: Docker Compose (example)}
\begin{lstlisting}[language=bash,caption={docker-compose.yml snippet}]
version: '3.8'
services:
  app:
    build: .
    ports:
      - "3000:3000"
    environment:
      - NODE_ENV=production
      - DB_HOST=db
      - DB_USER=root
      - DB_PASSWORD=password
      - DB_NAME=product_management_system
    depends_on:
      - db
    volumes:
      - ./uploads:/app/uploads

  db:
    image: mysql:8.0
    environment:
      - MYSQL_ROOT_PASSWORD=password
      - MYSQL_DATABASE=product_management_system
    ports:
      - "3306:3306"
    volumes:
      - mysql_data:/var/lib/mysql

volumes:
  mysql_data:
\end{lstlisting}

\section{B: Google App Engine (GCP) deploy snippet}
\begin{verbatim}
runtime: nodejs20
env_variables:
  NODE_ENV: production
  DB_HOST: /cloudsql/project-name:us-central1:product-management-db
  DB_USER: sigma
  DB_NAME: product_management_system
  DB_PASSWORD: "projects/project-name/secrets/db-password/versions/latest"
automatic_scaling:
  min_instances: 1
  max_instances: 10
\end{verbatim}

\section{C: Useful Links}
\begin{itemize}
  \item Docker Hub: \url{https://hub.docker.com/r/priyanshuksharma/project-management}
  \item Production App: \url{https://project-name.uc.r.appspot.com}
  \item Setup Endpoint: \url{https://project-name.uc.r.appspot.com/setup}
  \item Cleanup Endpoint: \url{https://project-name.uc.r.appspot.com/cleanup}
\end{itemize}

\section{D: Recent Updates (Latest Commit)}
\textbf{Authentication System Overhaul:}
\begin{itemize}
  \item Fixed admin user authentication for Google Cloud deployment
  \item Added cleanup route to remove old admin users from previous versions
  \item Enhanced admin user creation with proper error handling and existence checks
  \item Improved login debugging with comprehensive console logging
  \item Updated default admin credentials to: admin/admin123, sysadmin/Welcome@123, dbadmin/Welcome@123
\end{itemize}

\textbf{Registration System Updates:}
\begin{itemize}
  \item Commented out email pattern validation in registration form
  \item Email validation can be toggled by uncommenting pattern attribute
  \item Maintains backward compatibility with existing registration flow
  \item Allows registration with any email domain when pattern is disabled
\end{itemize}

\textbf{Google Cloud Platform Integration:}
\begin{itemize}
  \item Optimized for Google App Engine deployment
  \item Cloud SQL integration with socket path configuration
  \item Environment-specific database configuration
  \item Production-ready scaling and security settings
\end{itemize}

\textbf{Code Quality Improvements:}
\begin{itemize}
  \item Enhanced error handling in login authentication
  \item Better separation of user creation logic
  \item Improved database connection management
  \item Added comprehensive logging for troubleshooting
\end{itemize}

\section{E: Troubleshooting Authentication Issues}
\textbf{Common Login Problems:}
\begin{description}
  \item["Invalid username or password"] Check console logs for specific error. Verify user exists in database and password is correct.
  \item["Users found: 0"] User doesn't exist. Visit /setup endpoint to create admin users.
  \item["Password valid: false"] Password hash mismatch. Reset password or recreate user.
  \item[Database connection errors] Verify .env configuration and database service status.
\end{description}

\textbf{Debug Steps:}
\begin{enumerate}
  \item Check Google Cloud logs: \texttt{gcloud app logs tail -s default}
  \item Verify database connectivity and user existence
  \item Test password hashing with fresh hash comparison
  \item Ensure session configuration is correct
  \item Verify environment variables are properly set
\end{enumerate}

\chapter*{Document Information}
\addcontentsline{toc}{chapter}{Document Information}
\begin{tabular}{ll}
Version: & 1.2.0 \\
Last Updated: & \today \\
Prepared by: & MQI Internship Team  \\
Latest Changes: & Comprehensive GCP deployment guide, authentication system overhaul \\
Commit Hash: & e2a05ea (Fix admin user authentication and update registration form) \\
 \\
\end{tabular}

\vspace{1cm}
\textbf{Change Log:}
\begin{itemize}
  \item v1.2.0: Added detailed GCP deployment, authentication troubleshooting, admin management
  \item v1.1.0: Updated admin credentials, added cleanup procedures
  \item v1.0.0: Initial comprehensive deployment guide
\end{itemize}

\end{document}

